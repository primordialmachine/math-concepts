%%%%%%%%%%%%%%%%%%%%%%%%%%%%%%%%%%%%%%%%%%%%%%%%%%%%%%%%%%%%%%%%%%%%%%%%%%%%%%%%%%%%%%%%%%%%%%%%%%%
%
% Primordial Machine's Trigonometric Functors Library
% Copyright (C) 2017-2019 Michael Heilmann
%
% This software is provided 'as-is', without any express or implied warranty.
% In no event will the authors be held liable for any damages arising from the
% use of this software.
%
% Permission is granted to anyone to use this software for any purpose,
% including commercial applications, and to alter it and redistribute it
% freely, subject to the following restrictions:
%
% 1. The origin of this software must not be misrepresented;
%    you must not claim that you wrote the original software.
%    If you use this software in a product, an acknowledgment
%    in the product documentation would be appreciated but is not required.
%
% 2. Altered source versions must be plainly marked as such,
%    and must not be misrepresented as being the original software.
%
% 3. This notice may not be removed or altered from any source distribution.
%
%%%%%%%%%%%%%%%%%%%%%%%%%%%%%%%%%%%%%%%%%%%%%%%%%%%%%%%%%%%%%%%%%%%%%%%%%%%%%%%%%%%%%%%%%%%%%%%%%%%

\documentclass[oneside]{book}

\input{common}
\SetOrganization{Primordial Machine}
\SetLibraryName{Trigonometric Functors}
\SetLibraryVersion{1.1}
\SetLibraryRepository{https://github.com/primordialmachine/trigonometric-functors}
\SetAuthor{Michael Heilmann}
\SetEmail{michaelheilmann@primordialmachine.com}


\SetLibraryIncludeFileName{include.hpp}
\SetLibraryIncludesDirectoryPath{primordialmachine/trigonometric-functors/\newline\$(PlatformTarget.toLower())/\$(Configuration.toLower())/includes}

\SetLibraryIncludeDirectiveFilePath{primordialmachine/trigonometric\_functors/include.hpp}

\SetLibraryStaticLibrariesDirectoryPath{primordialmachine/trigonometric-functors/\newline\$(PlatformTarget.toLower())/\$(Configuration.toLower())/libraries}
\SetLibraryStaticLibraryFileName{trigonometric-functors.lib}

\SetDocumentType{User Manual}

\begin{document}

\frontmatter

\begin{titlepage}
\maketitle
\end{titlepage}

\tableofcontents
\addtocontents{toc}{\protect\thispagestyle{empty}}
\pagenumbering{gobble}

\mainmatter

\chapter{Synopsis}
C++ 17 library providing trigonometric functors.
The library is made available publicly on
\href{\GetLibraryRepository}{Github}
under the
\href{\GetLibraryRepository/blob/master/LICENSE}{MIT License}.

\chapter{Requirements}
If the library is added to a project, then one needs to add the Primordial Machine libraries
\href{https://github.com/primordialmachine/errors}{Errors}
and
\href{https://github.com/primordialmachine/functors}{Functors}
as well.

\chapter{Limitations and Restrictions}
The library officially only supports Visual Studio 2017 and Windows 10.

%\section{Introductory example}
%Examples are located in the \href{\GetLibraryRepository/blob/master/examples}{examples} directory.

\input{building_visual_studio_2017}

\chapter{Library Interface Documentation}

\section{\texttt{namespace primordialmachine}}
The namespace this library is adding its declarations/definitions to.
The added namespace elements are documented below.

\input{functors}

%%%%%%%%%%%%%%%%%%%%%%%%%%%%%%%%%%%%%%%%%%%%%%%%%%%%%%%%%%%%%%%%%%%%%%%%%%%%%%%%%%%%%%%%%%%%%%%%%%%%
\section{\texttt{sin\_functor} (struct)}
A \textit{UnaryFunctorBase} which computes the
sine
of an angle.
The first parameter is the angle, the return value the sine of that angle.

%\noindent{}This library provides specializations of this functor for all floating point types.
%The specializations assume the angle is measured in radians.\\

\noindent{}\textcolor{orange}{\textit{Defect: No error/exception specification is provided.}}

\subsection{\texttt{sin} (function)}
A function which returns the value of \texttt{primordialmachine::sin\_functor\textlangle T\textrangle}
for a value of type \texttt{T}.

\noindent{}Possible implementations
\begin{verbatim}
template<typename T>
auto sin(const T& v) -> decltype(sin_functor<T, void>()(v))
{ return sin_functor<T, void>()(v); }
\end{verbatim}

\noindent{}\textcolor{orange}{\textit{Defect: No error/exception specification is provided.}}

%%%%%%%%%%%%%%%%%%%%%%%%%%%%%%%%%%%%%%%%%%%%%%%%%%%%%%%%%%%%%%%%%%%%%%%%%%%%%%%%%%%%%%%%%%%%%%%%%%%%
\section{\texttt{cos\_functor} (struct)}
A \textit{UnaryFunctorBase} which computes the
cosine
of an angle.
The first parameter is the angle, the return value the cosine of that angle.

\noindent{}\textcolor{orange}{\textit{Defect: No error/exception specification is provided.}}

\subsection{\texttt{cos} (function)}
A function which returns the value of \texttt{primordialmachine::cos\_functor\textlangle T\textrangle}
for a value of type \texttt{T}.

\noindent{}Possible implementations
\begin{verbatim}
template<typename T>
auto cos(const T& v) -> decltype(cos_functor<T, void>()(v))
{ return cos_functor<T, void>()(v); }
\end{verbatim}

\noindent{}\textcolor{orange}{\textit{Defect: No error/exception specification is provided.}}

%%%%%%%%%%%%%%%%%%%%%%%%%%%%%%%%%%%%%%%%%%%%%%%%%%%%%%%%%%%%%%%%%%%%%%%%%%%%%%%%%%%%%%%%%%%%%%%%%%%%
\section{\texttt{tan\_functor} (struct)}
A \textit{UnaryFunctorBase} which computes the
tangens
of an angle.
The first parameter is the angle, the return value the tangens of that angle.

\noindent{}\textcolor{orange}{\textit{Defect: No error/exception specification is provided.}}

\subsection{\texttt{tan} (function)}
A function which returns the value of \texttt{primordialmachine::tan\_functor\textlangle T\textrangle}
for a value of type \texttt{T}.

\noindent{}A possible implementation is
\begin{verbatim}
template<typename T>
auto tan(const T& v) -> decltype(tan_functor<T>()(v))
{ return tan_functor<T>()(v); }
\end{verbatim}

\noindent{}\textcolor{orange}{\textit{Defect: No error/exception specification is provided.}}

%%%%%%%%%%%%%%%%%%%%%%%%%%%%%%%%%%%%%%%%%%%%%%%%%%%%%%%%%%%%%%%%%%%%%%%%%%%%%%%%%%%%%%%%%%%%%%%%%%%%
\section{\texttt{acos\_functor} (struct)}
A \textit{UnaryFunctorBase} which computes the
arccosine
of a value.
The first parameter is the value, the return value the arccosine of that value.

\noindent{}\textcolor{orange}{\textit{Defect: No error/exception specification is provided.}}

\subsection{\texttt{acos} (function)}
A function which returns the value of \texttt{primordialmachine::acos\_functor\textlangle T\textrangle}
for a value of type \texttt{T}.

\noindent{}A possible implementation is
\begin{verbatim}
template<typename T>
auto acos(const T& v) -> decltype(acos_functor<T, void>()(v))
{ return acos_functor<T, void>()(v); }
\end{verbatim}

\noindent{}\textcolor{orange}{\textit{Defect: No error/exception specification is provided.}}
%%%%%%%%%%%%%%%%%%%%%%%%%%%%%%%%%%%%%%%%%%%%%%%%%%%%%%%%%%%%%%%%%%%%%%%%%%%%%%%%%%%%%%%%%%%%%%%%%%%%
\section{\texttt{asin\_functor} (struct)}
A \textit{UnaryFunctorBase} which computes the
arcsine
of a value.
The first parameter is the value, the return value the arcsine of that value.

\noindent{}\textcolor{orange}{\textit{Defect: No error/exception specification is provided.}}

\subsection{\texttt{asin} (function)}
A function which returns the value of \texttt{primordialmachine::asin\_functor\textlangle T\textrangle}
for a value of type \texttt{T}.

\noindent{}A possible implementation is
\begin{verbatim}
template<typename T>
auto asin(const T& v) -> decltype(asin_functor<T, void>()(v))
{ return asin_functor<T, void>()(v); }
\end{verbatim}

\noindent{}\textcolor{orange}{\textit{Defect: No error/exception specification is provided.}}
%%%%%%%%%%%%%%%%%%%%%%%%%%%%%%%%%%%%%%%%%%%%%%%%%%%%%%%%%%%%%%%%%%%%%%%%%%%%%%%%%%%%%%%%%%%%%%%%%%%%
\section{\texttt{cot\_functor} (struct)}
A \textit{UnaryFunctorBase} which computes the
cotangens of an angle.
The first parameter is the angle, the return value the cotangens of that angle.

\noindent{}\textcolor{orange}{\textit{Defect: No error/exception specification is provided.}}

\subsection{\texttt{cot} (function)}
A function which returns the value of \texttt{primordialmachine::cot\_functor\textlangle T\textrangle}
for a value of type \texttt{T}.

\noindent{}A possible implementation is
\begin{verbatim}
template<typename T>
auto cot(const T& v) -> decltype(cot_functor<T, void>()(v))
{ return cot_functor<T, void>()(v); }
\end{verbatim}

\noindent{}\textcolor{orange}{\textit{Defect: No error/exception specification is provided.}}

\end{document}
